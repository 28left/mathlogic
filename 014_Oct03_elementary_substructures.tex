% Options for packages loaded elsewhere
% Options for packages loaded elsewhere
\PassOptionsToPackage{unicode}{hyperref}
\PassOptionsToPackage{hyphens}{url}
\PassOptionsToPackage{dvipsnames,svgnames,x11names}{xcolor}
%
\documentclass[
]{article}
\usepackage{xcolor}
\usepackage[margin=0.8in]{geometry}
\usepackage{amsmath,amssymb}
\setcounter{secnumdepth}{-\maxdimen} % remove section numbering
\usepackage{iftex}
\ifPDFTeX
  \usepackage[T1]{fontenc}
  \usepackage[utf8]{inputenc}
  \usepackage{textcomp} % provide euro and other symbols
\else % if luatex or xetex
  \usepackage{unicode-math} % this also loads fontspec
  \defaultfontfeatures{Scale=MatchLowercase}
  \defaultfontfeatures[\rmfamily]{Ligatures=TeX,Scale=1}
\fi
\usepackage{lmodern}
\ifPDFTeX\else
  % xetex/luatex font selection
\fi
% Use upquote if available, for straight quotes in verbatim environments
\IfFileExists{upquote.sty}{\usepackage{upquote}}{}
\IfFileExists{microtype.sty}{% use microtype if available
  \usepackage[]{microtype}
  \UseMicrotypeSet[protrusion]{basicmath} % disable protrusion for tt fonts
}{}
\makeatletter
\@ifundefined{KOMAClassName}{% if non-KOMA class
  \IfFileExists{parskip.sty}{%
    \usepackage{parskip}
  }{% else
    \setlength{\parindent}{0pt}
    \setlength{\parskip}{6pt plus 2pt minus 1pt}}
}{% if KOMA class
  \KOMAoptions{parskip=half}}
\makeatother
% Make \paragraph and \subparagraph free-standing
\makeatletter
\ifx\paragraph\undefined\else
  \let\oldparagraph\paragraph
  \renewcommand{\paragraph}{
    \@ifstar
      \xxxParagraphStar
      \xxxParagraphNoStar
  }
  \newcommand{\xxxParagraphStar}[1]{\oldparagraph*{#1}\mbox{}}
  \newcommand{\xxxParagraphNoStar}[1]{\oldparagraph{#1}\mbox{}}
\fi
\ifx\subparagraph\undefined\else
  \let\oldsubparagraph\subparagraph
  \renewcommand{\subparagraph}{
    \@ifstar
      \xxxSubParagraphStar
      \xxxSubParagraphNoStar
  }
  \newcommand{\xxxSubParagraphStar}[1]{\oldsubparagraph*{#1}\mbox{}}
  \newcommand{\xxxSubParagraphNoStar}[1]{\oldsubparagraph{#1}\mbox{}}
\fi
\makeatother


\usepackage{longtable,booktabs,array}
\usepackage{calc} % for calculating minipage widths
% Correct order of tables after \paragraph or \subparagraph
\usepackage{etoolbox}
\makeatletter
\patchcmd\longtable{\par}{\if@noskipsec\mbox{}\fi\par}{}{}
\makeatother
% Allow footnotes in longtable head/foot
\IfFileExists{footnotehyper.sty}{\usepackage{footnotehyper}}{\usepackage{footnote}}
\makesavenoteenv{longtable}
\usepackage{graphicx}
\makeatletter
\newsavebox\pandoc@box
\newcommand*\pandocbounded[1]{% scales image to fit in text height/width
  \sbox\pandoc@box{#1}%
  \Gscale@div\@tempa{\textheight}{\dimexpr\ht\pandoc@box+\dp\pandoc@box\relax}%
  \Gscale@div\@tempb{\linewidth}{\wd\pandoc@box}%
  \ifdim\@tempb\p@<\@tempa\p@\let\@tempa\@tempb\fi% select the smaller of both
  \ifdim\@tempa\p@<\p@\scalebox{\@tempa}{\usebox\pandoc@box}%
  \else\usebox{\pandoc@box}%
  \fi%
}
% Set default figure placement to htbp
\def\fps@figure{htbp}
\makeatother





\setlength{\emergencystretch}{3em} % prevent overfull lines

\providecommand{\tightlist}{%
  \setlength{\itemsep}{0pt}\setlength{\parskip}{0pt}}



 


\makeatletter
\@ifpackageloaded{caption}{}{\usepackage{caption}}
\AtBeginDocument{%
\ifdefined\contentsname
  \renewcommand*\contentsname{Table of contents}
\else
  \newcommand\contentsname{Table of contents}
\fi
\ifdefined\listfigurename
  \renewcommand*\listfigurename{List of Figures}
\else
  \newcommand\listfigurename{List of Figures}
\fi
\ifdefined\listtablename
  \renewcommand*\listtablename{List of Tables}
\else
  \newcommand\listtablename{List of Tables}
\fi
\ifdefined\figurename
  \renewcommand*\figurename{Figure}
\else
  \newcommand\figurename{Figure}
\fi
\ifdefined\tablename
  \renewcommand*\tablename{Table}
\else
  \newcommand\tablename{Table}
\fi
}
\@ifpackageloaded{float}{}{\usepackage{float}}
\floatstyle{ruled}
\@ifundefined{c@chapter}{\newfloat{codelisting}{h}{lop}}{\newfloat{codelisting}{h}{lop}[chapter]}
\floatname{codelisting}{Listing}
\newcommand*\listoflistings{\listof{codelisting}{List of Listings}}
\usepackage{amsthm}
\theoremstyle{definition}
\newtheorem{exercise}{Exercise}[section]
\theoremstyle{plain}
\newtheorem{theorem}{Theorem}[section]
\theoremstyle{remark}
\AtBeginDocument{\renewcommand*{\proofname}{Proof}}
\newtheorem*{remark}{Remark}
\newtheorem*{solution}{Solution}
\newtheorem{refremark}{Remark}[section]
\newtheorem{refsolution}{Solution}[section]
\makeatother
\makeatletter
\makeatother
\makeatletter
\@ifpackageloaded{caption}{}{\usepackage{caption}}
\@ifpackageloaded{subcaption}{}{\usepackage{subcaption}}
\makeatother
\usepackage{bookmark}
\IfFileExists{xurl.sty}{\usepackage{xurl}}{} % add URL line breaks if available
\urlstyle{same}
\hypersetup{
  pdftitle={Math 557 Oct 3},
  colorlinks=true,
  linkcolor={blue},
  filecolor={Maroon},
  citecolor={Blue},
  urlcolor={Blue},
  pdfcreator={LaTeX via pandoc}}


\title{Math 557 Oct 3}
\author{}
\date{}
\begin{document}
\maketitle


\section{Elementary Substructures}\label{elementary-substructures}

\subsection{Key Concepts}\label{key-concepts}

\begin{itemize}
\tightlist
\item
  Two \(\mathcal{L}\)-structures are \textbf{elementary equivalent} if
  for all \(\mathcal{L}\)-sentences \(\sigma\),
  \[\mathcal{M} \models \sigma \; \iff \; \mathcal{N} \models \sigma\]
\end{itemize}

\subsection{Problems}\label{problems}

\begin{exercise}[]\protect\hypertarget{exr-}{}\label{exr-}

\hfill\break
We have the following relations between structures
\(\mathcal{M}, \mathcal{N}\):

\[\subseteq, \preceq, \equiv, \cong\]

Draw a diagram indicating implications between these relations, giving
counterexamples if one relation does not imply another.

\end{exercise}

\subsubsection{Tarski-Vaught test}\label{tarski-vaught-test}

\begin{exercise}[]\protect\hypertarget{exr-}{}\label{exr-}

\hfill\break
\textbf{THM}: Suppose \(\mathcal{M} \subseteq \mathcal{N}\) and that for
any formula \(\psi(x, \vec{y})\) and any \(\vec{a} in M\), if there
exists \(b \in N\) such that \(\mathcal{N} \models \psi[b, \vec{a}]\),
then there also exists \(c \in M\) such that
\(\mathcal{N} \models \psi[c, \vec{a}]\). Then we have
\(\mathcal{M} \preceq \mathcal{N}\).

Prove this theorem by induction in \(\operatorname{ht}(\psi)\).

\emph{Before you start, which inductive case do you think will require
the most work?}

\end{exercise}

As an application of the Tarski-Vaught test, we get another criterion
for \(\preceq\) using automorphisms of the bigger structure.

:::\\
Suppose \(\mathcal{M} \subseteq \mathcal{N}\) and that for any finite
subset \(A \subeteq M\) and \(b \in N\), there exists an automorphism of
\(\mathcal{N}\) that fixes \(A\) pointwise and maps \(b\) into \(M\).
Show that \$\(\mathcal{M} \preceq \mathcal{N}\). :::

:::\\
Use the previous criterion to show that \[
(\mathbb{Q}, <) \preceq (\mathbb{R}, <)
\] :::

\subsubsection{More on DLOs}\label{more-on-dlos}

We have seen previously that the theory \(\operatorname{DLO}\) is
\emph{\(\aleph_0\)-categorical}, i.e., there is only one countable model
up to isomorphism.

We will now see that this actually implies \(\operatorname{DLO}\) is
complete.

We need the following theorem which we will be an easy consequence of
the Löwenheim-SKolem Theorems we will prove next week.

\begin{theorem}[]\protect\hypertarget{thm-}{}\label{thm-}

Let \(T\) be an \(\mathcal{L}\)-theory that has an infinite model. If
\(\kappa\) is an infinite cardinal and \(\kappa \geq |\mathcal{L}|\),
then there is a model of \(T\) of cardinality \(\kappa\).

\end{theorem}

\begin{exercise}[Vaught's test]\protect\hypertarget{exr-}{}\label{exr-}

Suppose \(T\) is a consistent \(\mathcal{L}\)-theory with no finite
models. If \(T\) is \(\kappa\)-categorical for some
\(\kappa \geq |\mathcal{L}|\), then \(T\) is complete.

\end{exercise}




\end{document}
