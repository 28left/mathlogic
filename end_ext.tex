Von besonderer Bedeutung ist, dass $\Delta_0$-Formeln in einer Struktur dasselbe bedeuten wie in allen Enderweiterungen:

\section{Erhaltungseigenschaften unter Enderweiterungen}\label{erhaltung}
$\mathcal{N},  \mathcal{M}$ \emph{seien Strukturen der Sprache L von} $PA^-, \mathcal{N} \subseteq_{end} \mathcal{M},  \vec{a} \in N$. \emph{Dann gilt:}
\begin{enumerate}
\item[(i)] \emph{jede} $\Delta_0$\emph{-Formel} $\varphi(\vec{v})$ \emph{ist absolut:}
\begin{equation*}
\mathcal{N}\models \varphi[\vec{a}] \iff \mathcal{M}\models \varphi[\vec{a}], 
\end{equation*}
\item[(ii)] \emph{jede} $\Sigma_1$\emph{-Formel} $\varphi(\vec{v})$ \emph{ist aufwärts-persistent:}
\begin{equation*}
\mathcal{N}\models \varphi[\vec{a}] \Longrightarrow \mathcal{M}\models \varphi[\vec{a}], 
\end{equation*}
\item[(iii)] \emph{jede} $\Pi_1$\emph{-Formel} $\varphi(\vec{v})$ \emph{ist abwärts-persistent:}
\begin{equation*}
\mathcal{M}\models \varphi[\vec{a}] \Longrightarrow \mathcal{N}\models \varphi[\vec{a}],
\end{equation*}
\item[(iv)] \emph{jede} $\Delta_1$\emph{-Formel} $\varphi(\vec{v})$ \emph{ist absolut:}
\begin{equation*}
\mathcal{N}\models \varphi[\vec{a}] \iff \mathcal{M}\models \varphi[\vec{a}].
\end{equation*} 
\end{enumerate}

\begin{proof}von (i) durch Induktion über den Formelaufbau von $\varphi(\vec{v})$, wobei nur der Fall eines beschränkten Quantors von Interesse ist. Da aber  $\mathcal{M}$ Enderweiterung von $\mathcal{N}$ ist, werden von $M$ unterhalb eines Elementes von $N$ keine neuen Elemente eingefügt, so dass ein beschränkter Quantor in beiden Strukturen dasselbe besagt.
\end{proof}

Es sei $\; \Sigma_1$-$Th(\mathbb{N}):= \{\sigma |  \sigma \; \Sigma_1\text{-Satz mit} \; \mathbb{N} \models \sigma \}$. Dann gilt:

\subsection*{Korollar}
\begin{equation*}
PA^- \models \Sigma_1\text{-}Th(\mathbb{N})
\end{equation*}
 \begin{Beweis}
Es sei $\mathcal{N} \models PA^-$. Nach kann \ref{enderweiterung} man annehmen, dass $\mathbb{N} \subseteq_{end} \mathcal{N}$, und die Behauptung folgt dann aus \ref{erhaltung} (ii).
\end{Beweis}

Somit kann man in der Theorie $PA^-$ alle $\Sigma_1$-Sätze beweisen, die im Standardmodell gelten, aber bereits nicht mehr alle wahren $\Pi_1$-Sätze, denn der $\Pi_1$-Satz, welcher besagt, dass jede Zahl gerade oder ungerade ist:
\begin{equation*}
(*) \quad \forall x \;  \exists y \le x \; (x = 2 \cdot y \vee x = 2 \cdot y +1)
\end{equation*}
ist im Standardmodell wahr, nicht aber in dem 
$PA^-$-Modell $\mathbb{Z}[X]^+$. Bereits wahre $\forall$-Sätze (d.h. Sätze der Form $\forall \vec{x} \; \psi$ mit \emph{quantorenfreiem} $\psi$), die im Standardmodell gelten, brauchen nicht in $PA^-$ beweisbar zu sein, z.\;B. ist auch der  $\forall$-Satz
\begin{equation*}
\forall x,y  (x^2 \ne 2 \cdot y^2)
\end{equation*}
im Standardmodell wahr, nicht aber in $PA^-$ beweisbar ($\mathbb{Z}/(X^2 - 2Y^2)$ ist ein Gegenbeispiel). Somit ist der obige Satz (*) ein Beispiel für einen $\Pi_1$-Satz, der kein $\forall$-Satz ist, wo man also auf den beschränkten Quantor nicht verzichten kann! (Übrigens kann man leicht zeigen, dass $\mathbb{Z}[X]^+$ immerhin ein Modell aller $\forall$-Sätze ist, die im Standardmodell gelten.)
