% Options for packages loaded elsewhere
% Options for packages loaded elsewhere
\PassOptionsToPackage{unicode}{hyperref}
\PassOptionsToPackage{hyphens}{url}
\PassOptionsToPackage{dvipsnames,svgnames,x11names}{xcolor}
%
\documentclass[
]{article}
\usepackage{xcolor}
\usepackage[margin=0.8in]{geometry}
\usepackage{amsmath,amssymb}
\setcounter{secnumdepth}{-\maxdimen} % remove section numbering
\usepackage{iftex}
\ifPDFTeX
  \usepackage[T1]{fontenc}
  \usepackage[utf8]{inputenc}
  \usepackage{textcomp} % provide euro and other symbols
\else % if luatex or xetex
  \usepackage{unicode-math} % this also loads fontspec
  \defaultfontfeatures{Scale=MatchLowercase}
  \defaultfontfeatures[\rmfamily]{Ligatures=TeX,Scale=1}
\fi
\usepackage{lmodern}
\ifPDFTeX\else
  % xetex/luatex font selection
\fi
% Use upquote if available, for straight quotes in verbatim environments
\IfFileExists{upquote.sty}{\usepackage{upquote}}{}
\IfFileExists{microtype.sty}{% use microtype if available
  \usepackage[]{microtype}
  \UseMicrotypeSet[protrusion]{basicmath} % disable protrusion for tt fonts
}{}
\makeatletter
\@ifundefined{KOMAClassName}{% if non-KOMA class
  \IfFileExists{parskip.sty}{%
    \usepackage{parskip}
  }{% else
    \setlength{\parindent}{0pt}
    \setlength{\parskip}{6pt plus 2pt minus 1pt}}
}{% if KOMA class
  \KOMAoptions{parskip=half}}
\makeatother
% Make \paragraph and \subparagraph free-standing
\makeatletter
\ifx\paragraph\undefined\else
  \let\oldparagraph\paragraph
  \renewcommand{\paragraph}{
    \@ifstar
      \xxxParagraphStar
      \xxxParagraphNoStar
  }
  \newcommand{\xxxParagraphStar}[1]{\oldparagraph*{#1}\mbox{}}
  \newcommand{\xxxParagraphNoStar}[1]{\oldparagraph{#1}\mbox{}}
\fi
\ifx\subparagraph\undefined\else
  \let\oldsubparagraph\subparagraph
  \renewcommand{\subparagraph}{
    \@ifstar
      \xxxSubParagraphStar
      \xxxSubParagraphNoStar
  }
  \newcommand{\xxxSubParagraphStar}[1]{\oldsubparagraph*{#1}\mbox{}}
  \newcommand{\xxxSubParagraphNoStar}[1]{\oldsubparagraph{#1}\mbox{}}
\fi
\makeatother


\usepackage{longtable,booktabs,array}
\usepackage{calc} % for calculating minipage widths
% Correct order of tables after \paragraph or \subparagraph
\usepackage{etoolbox}
\makeatletter
\patchcmd\longtable{\par}{\if@noskipsec\mbox{}\fi\par}{}{}
\makeatother
% Allow footnotes in longtable head/foot
\IfFileExists{footnotehyper.sty}{\usepackage{footnotehyper}}{\usepackage{footnote}}
\makesavenoteenv{longtable}
\usepackage{graphicx}
\makeatletter
\newsavebox\pandoc@box
\newcommand*\pandocbounded[1]{% scales image to fit in text height/width
  \sbox\pandoc@box{#1}%
  \Gscale@div\@tempa{\textheight}{\dimexpr\ht\pandoc@box+\dp\pandoc@box\relax}%
  \Gscale@div\@tempb{\linewidth}{\wd\pandoc@box}%
  \ifdim\@tempb\p@<\@tempa\p@\let\@tempa\@tempb\fi% select the smaller of both
  \ifdim\@tempa\p@<\p@\scalebox{\@tempa}{\usebox\pandoc@box}%
  \else\usebox{\pandoc@box}%
  \fi%
}
% Set default figure placement to htbp
\def\fps@figure{htbp}
\makeatother





\setlength{\emergencystretch}{3em} % prevent overfull lines

\providecommand{\tightlist}{%
  \setlength{\itemsep}{0pt}\setlength{\parskip}{0pt}}



 


\makeatletter
\@ifpackageloaded{caption}{}{\usepackage{caption}}
\AtBeginDocument{%
\ifdefined\contentsname
  \renewcommand*\contentsname{Table of contents}
\else
  \newcommand\contentsname{Table of contents}
\fi
\ifdefined\listfigurename
  \renewcommand*\listfigurename{List of Figures}
\else
  \newcommand\listfigurename{List of Figures}
\fi
\ifdefined\listtablename
  \renewcommand*\listtablename{List of Tables}
\else
  \newcommand\listtablename{List of Tables}
\fi
\ifdefined\figurename
  \renewcommand*\figurename{Figure}
\else
  \newcommand\figurename{Figure}
\fi
\ifdefined\tablename
  \renewcommand*\tablename{Table}
\else
  \newcommand\tablename{Table}
\fi
}
\@ifpackageloaded{float}{}{\usepackage{float}}
\floatstyle{ruled}
\@ifundefined{c@chapter}{\newfloat{codelisting}{h}{lop}}{\newfloat{codelisting}{h}{lop}[chapter]}
\floatname{codelisting}{Listing}
\newcommand*\listoflistings{\listof{codelisting}{List of Listings}}
\usepackage{amsthm}
\theoremstyle{definition}
\newtheorem{exercise}{Exercise}[section]
\theoremstyle{plain}
\newtheorem{theorem}{Theorem}[section]
\theoremstyle{plain}
\newtheorem{lemma}{Lemma}[section]
\theoremstyle{remark}
\AtBeginDocument{\renewcommand*{\proofname}{Proof}}
\newtheorem*{remark}{Remark}
\newtheorem*{solution}{Solution}
\newtheorem{refremark}{Remark}[section]
\newtheorem{refsolution}{Solution}[section]
\makeatother
\makeatletter
\makeatother
\makeatletter
\@ifpackageloaded{caption}{}{\usepackage{caption}}
\@ifpackageloaded{subcaption}{}{\usepackage{subcaption}}
\makeatother
\usepackage{bookmark}
\IfFileExists{xurl.sty}{\usepackage{xurl}}{} % add URL line breaks if available
\urlstyle{same}
\hypersetup{
  pdftitle={Math 557 Oct 8},
  colorlinks=true,
  linkcolor={blue},
  filecolor={Maroon},
  citecolor={Blue},
  urlcolor={Blue},
  pdfcreator={LaTeX via pandoc}}


\title{Math 557 Oct 8}
\author{}
\date{}
\begin{document}
\maketitle


\section{Ultraproducts}\label{ultraproducts}

\subsection{Direct Products}\label{direct-products}

Let \((\mathcal{M}_i)_{i \in I}\) be a family of \(L\)-structures.

We define the \textbf{direct product} \[
\mathcal{M} = \prod_{i \in I} \mathcal{M}_i
\] as follows:

\begin{enumerate}
\def\labelenumi{\arabic{enumi}.}
\tightlist
\item
  The universe is the Cartesian product \(M = \prod_{i \in I} M_i\). If
  \(a\) is an element of \(M\), we denote its \(i\)-th component (an
  element of \(M_i\)) by \(a_i\) and extend this notation to vectors: if
  \(\vec a\) is a finite tuple in \(M^n\), \(\vec a_i\) denotes the
  \(n\)-tuple in \(M_i\) consisting of the \(M_i\)-entries of
  \(\vec a\).
\item
  For each relation symbol \(R \in \mathcal{L}\), \[
  R^{\mathcal{M}}(\vec a) :\iff \forall i \in I,\,
  \vec a_i \in R^{\mathcal{M}_i}
  \]
\item
  For each function symbol \(f \in \mathcal{L}\), \[
  f^{\mathcal{M}}(\vec a)
  := (f^{\mathcal{M}_i}(\vec a_i))_{i\in I}.
  \]
\item
  For each constant \(c \in \mathcal{L}\), \[
  c^{\mathcal{M}} = (c^{\mathcal{M}_i})_{i\in I}.
  \]
\end{enumerate}

\subsubsection{Examples and
Observations}\label{examples-and-observations}

\begin{itemize}
\tightlist
\item
  The direct product of groups is again a group (componentwise
  operation).
\item
  The direct product of fields is \textbf{not} a field: \[
  (1,0)\cdot(0,1) = (0,0).
  \]
\item
  The direct product of linear orders is only a \textbf{partial order}.
\end{itemize}

We often want to preserve properties that hold in ``most'' component
structures.\\
To formalize ``most,'' we use \textbf{filters} on \(I\).

\subsection{Filters and Ultrafilters}\label{filters-and-ultrafilters}

A \textbf{filter} \(\mathcal{F}\) on a set \(I\) is a nonempty
collection of subsets of \(I\) satisfying:

\begin{enumerate}
\def\labelenumi{\arabic{enumi}.}
\tightlist
\item
  \(\emptyset \notin \mathcal{F}\)\\
\item
  If \(A,B \in \mathcal{F}\), then \(A \cap B \in \mathcal{F}\)\\
\item
  If \(A \in \mathcal{F}\) and \(A \subseteq B \subseteq I\), then
  \(B \in \mathcal{F}\)
\end{enumerate}

An \textbf{ultrafilter} \(\mathcal{U}\) is a maximal filter,
equivalently:

\begin{quote}
For all \(A \subseteq I\), either \(A \in \mathcal{U}\) or
\(I \setminus A \in \mathcal{U}\).
\end{quote}

Ultrafilters interact nicely with logical operators:

\begin{itemize}
\tightlist
\item
  \(A \not \in \mathcal{U} \iff I \backslash A \in \mathcal{U}\),
\item
  \(A  \in \mathcal{U} \wedge B \in \mathcal{U} \iff A \cap B \in \mathcal{U}\),
\item
  \(A  \in \mathcal{U} \vee B \in \mathcal{U} \iff A \cup B \in \mathcal{U}\).
\end{itemize}

\subsubsection{Examples}\label{examples}

\begin{itemize}
\item
  A \textbf{principal filter} is of the form\\
  \[
  \mathcal{F}_A = \{ X \subseteq I : A \subseteq X \}
  \] for some nonempty \(A \subseteq I\).\\
  If \(A = \{a\}\), then \(\mathcal{F}_A\) is a \textbf{principal
  ultrafilter}.
\item
  A \textbf{free} (non-principal) ultrafilter exists on every infinite
  set \(I\)\\
  (via Zorn's Lemma / Boolean prime ideal theorem).
\end{itemize}

\subsubsection{Existence of
Ultrafilters}\label{existence-of-ultrafilters}

A family of sets has the \textbf{finite intersection property (FIP)} if
every finite subfamily has nonempty intersection.

\begin{theorem}[]\protect\hypertarget{thm-uf-existence}{}\label{thm-uf-existence}

\hfill\break
If a family \(\mathcal{A} \subseteq \mathcal{P}(I)\) has the FIP,\\
then there exists an ultrafilter \(\mathcal{U}\) on \(I\) with
\(\mathcal{A} \subseteq \mathcal{U}\).

\end{theorem}

\subsection{Reduced Products}\label{reduced-products}

Given a filter \(\mathcal{F}\) on \(I\) and structures
\((\mathcal{M}_i)_{i \in I}\), define the \textbf{reduced product} \[
\mathcal{M} / \mathcal{F}
\] as follows.

Let \(M = \prod_{i \in I} M_i\). For \(a,b \in M\), define \[
a \sim_{\mathcal{F}} b \iff \{\, i \in I : a_i = b_i \,\} \in \mathcal{F}.
\]

The universe of \(\mathcal{M}/\mathcal{F}\) is the quotient
\(M / {\sim_{\mathcal{F}}}\), with elements denoted \(a_{\mathcal{F}}\)
(alternatively, \(a/\mathcal{F}\)).

For symbols of \(\mathcal{L}\):

\begin{itemize}
\item
  \textbf{Relations:}\\
  \[
  R^{\mathcal{M}/\mathcal{F}}(\vec a_{\mathcal{F}})
   : \iff  \{\, i : \mathcal{M}_i \models R(\vec a_i) \,\} \in \mathcal{F}.
  \]
\item
  \textbf{Functions:}\\
  \[
  f^{\mathcal{M}/\mathcal{F}}(\vec a_{\mathcal{F}}) =
  [\, (f^{\mathcal{M}_i}(\vec a_i))_{i \in I} \,]_{\mathcal{F}}.
  \]
\item
  \textbf{Constants:}\\
  \[
  c^{\mathcal{M}/\mathcal{F}} = ((c^{\mathcal{M}_i})_{i\in I})_{\mathcal{F}}.
  \]
\end{itemize}

\begin{exercise}[]\protect\hypertarget{exr-}{}\label{exr-}

Check that the above definition does not depend on the choice of
representative for each equivalence class

\end{exercise}

\subsubsection{Ultraproducts}\label{ultraproducts-1}

If \(\mathcal{U}\) is an \textbf{ultrafilter} on \(I\), the reduced
product \[
\prod_{i \in I} \mathcal{M}_i / \mathcal{U}
\] is called the \textbf{ultraproduct} of \((\mathcal{M}_i)_{i \in I}\)
modulo \(\mathcal{U}\).

When all \(\mathcal{M}_i\) are the same structure \(\mathcal{M}\), we
get an \textbf{ultrapower} \[
\mathcal{M}^I / \mathcal{U}.
\]

\subsection{Łoś' Theorem}\label{ux142oux15b-theorem}

Let \(\mathcal{M} = \prod_{i \in I} \mathcal{M}_i / \mathcal{U}\) be an
ultraproduct.

\begin{theorem}[]\protect\hypertarget{thm-los}{}\label{thm-los}

\hfill\break
For every \(\mathcal{L}\)-formula \(\varphi(x_1,\dots,x_n)\) and
tuples\\
\(\vec a \in \prod_{i \in I} M_i\), \[
\mathcal{M} \models \varphi[\vec a_{\mathcal{U}}]
\iff
\{\, i \in I : \mathcal{M}_i \models \varphi[\vec a_i]\,\} \in \mathcal{U}.
\]

\end{theorem}

For any \(\mathcal{L}\)-formula \(\varphi(v_0,\ldots,v_{n-1})\) a a
tuple \(\vec{a} \in \prod M_i\) we define the \emph{Boolean extension}
as \[
\|\varphi(\vec{a}) \| := \{i \in I| \mathcal{M}_i \models \varphi[\vec{a}_i]\} 
\]

\begin{lemma}[]\protect\hypertarget{lem-}{}\label{lem-}

~

\begin{enumerate}
\def\labelenumi{\arabic{enumi}.}
\tightlist
\item
  \(\| \neg \varphi(\vec{a}) \| = I \backslash \|\varphi(\vec{a}) \|\),
\item
  \(\| (\varphi \wedge \psi)(\vec{a}) \| =\| \varphi(\vec{a}) \| \cap \| \psi(\vec{a}) \|\),
\item
  \(\| (\varphi \vee \psi)(\vec{a}) \| =\| \varphi(\vec{a}) \| \cup \| \psi(\vec{a}) \|\),
\item
  For all tuples \(\vec{a}\) and elements \(b\) in \(A\): \[
  \|\varphi(\vec{a},b) \| \subseteq \| (\exists v_n \varphi)(\vec{a}) \|,
  \] and there exists \(b \in M\) such that \[
  \|\varphi(\vec{a},b) \| = \| (\exists v_n \varphi)(\vec{a}) \|.
  \]
\end{enumerate}

\end{lemma}

\section{Proof of Łoś's Theorem
(Sketch)}\label{proof-of-ux142oux15bs-theorem-sketch}

We prove by induction on the structure of \(\varphi\).

\begin{enumerate}
\def\labelenumi{\arabic{enumi}.}
\item
  \textbf{Atomic formulas:}\\
  True by the definitions of \(R^{\mathcal{M}}\) and
  \(f^{\mathcal{M}}\).
\item
  \textbf{Boolean connectives:}\\
  The ultrafilter properties ensure

  \begin{itemize}
  \tightlist
  \item
    \(\{ i : \mathcal{M}_i \models \neg\psi(\bar a_i)\} = I \setminus \{ i : \mathcal{M}_i \models \psi(\bar a_i)\}\)\\
  \item
    \(\{ i : \mathcal{M}_i \models (\psi \wedge \theta)(\bar a_i)\}
     = \{ i : \mathcal{M}_i \models \psi(\bar a_i)\}
       \cap \{ i : \mathcal{M}_i \models \theta(\bar a_i)\}\)\\
    and ultrafilters are closed under these operations.
  \end{itemize}
\item
  \textbf{Existential quantifiers:}\\
  Suppose \[
  \mathcal{M} \models \exists y\, \psi(y,\bar a).
  \] Then for each \(i\), pick \(b_i\) such that
  \(\mathcal{M}_i \models \psi(b_i,\bar a_i)\) whenever possible, and
  set \(b = [b_i]_{\mathcal{U}}\). By definition of relations in the
  ultraproduct, we get \(\mathcal{M} \models \psi(b,\bar a)\).
\end{enumerate}

Thus the statement holds for all formulas.

\section{Applications}\label{applications}

\begin{itemize}
\item
  \textbf{Preservation of theories:}\\
  If each \(\mathcal{M}_i \models T\), then
  \(\prod_i \mathcal{M}_i / \mathcal{U} \models T\).
\item
  \textbf{Ultrapowers:}\\
  \(\mathcal{M} \equiv \mathcal{M}^I / \mathcal{U}\), i.e., a structure
  is elementarily equivalent to any of its ultrapowers.
\item
  \textbf{Nonstandard models:}\\
  For example, the ultrapower \(\mathbb{N}^\mathbb{N}/\mathcal{U}\)
  (with \(\mathcal{U}\) non-principal) yields a countably saturated
  nonstandard model of arithmetic.
\end{itemize}

\section{Remarks}\label{remarks}

\begin{itemize}
\item
  Ultraproducts provide a bridge between algebraic and logical
  constructions.\\
  Algebraically they are products modulo a maximal ideal
  (ultrafilter).\\
  Logically they yield the most powerful preservation theorem for
  first-order truth.
\item
  They also give elegant proofs of the \textbf{Compactness Theorem} and
  \textbf{Completeness Theorem}: every finitely satisfiable theory has a
  model that can be constructed as an ultraproduct of its finite
  fragments.
\end{itemize}




\end{document}
