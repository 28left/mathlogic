% Options for packages loaded elsewhere
% Options for packages loaded elsewhere
\PassOptionsToPackage{unicode}{hyperref}
\PassOptionsToPackage{hyphens}{url}
\PassOptionsToPackage{dvipsnames,svgnames,x11names}{xcolor}
%
\documentclass[
]{article}
\usepackage{xcolor}
\usepackage[margin=0.8in]{geometry}
\usepackage{amsmath,amssymb}
\setcounter{secnumdepth}{-\maxdimen} % remove section numbering
\usepackage{iftex}
\ifPDFTeX
  \usepackage[T1]{fontenc}
  \usepackage[utf8]{inputenc}
  \usepackage{textcomp} % provide euro and other symbols
\else % if luatex or xetex
  \usepackage{unicode-math} % this also loads fontspec
  \defaultfontfeatures{Scale=MatchLowercase}
  \defaultfontfeatures[\rmfamily]{Ligatures=TeX,Scale=1}
\fi
\usepackage{lmodern}
\ifPDFTeX\else
  % xetex/luatex font selection
\fi
% Use upquote if available, for straight quotes in verbatim environments
\IfFileExists{upquote.sty}{\usepackage{upquote}}{}
\IfFileExists{microtype.sty}{% use microtype if available
  \usepackage[]{microtype}
  \UseMicrotypeSet[protrusion]{basicmath} % disable protrusion for tt fonts
}{}
\makeatletter
\@ifundefined{KOMAClassName}{% if non-KOMA class
  \IfFileExists{parskip.sty}{%
    \usepackage{parskip}
  }{% else
    \setlength{\parindent}{0pt}
    \setlength{\parskip}{6pt plus 2pt minus 1pt}}
}{% if KOMA class
  \KOMAoptions{parskip=half}}
\makeatother
% Make \paragraph and \subparagraph free-standing
\makeatletter
\ifx\paragraph\undefined\else
  \let\oldparagraph\paragraph
  \renewcommand{\paragraph}{
    \@ifstar
      \xxxParagraphStar
      \xxxParagraphNoStar
  }
  \newcommand{\xxxParagraphStar}[1]{\oldparagraph*{#1}\mbox{}}
  \newcommand{\xxxParagraphNoStar}[1]{\oldparagraph{#1}\mbox{}}
\fi
\ifx\subparagraph\undefined\else
  \let\oldsubparagraph\subparagraph
  \renewcommand{\subparagraph}{
    \@ifstar
      \xxxSubParagraphStar
      \xxxSubParagraphNoStar
  }
  \newcommand{\xxxSubParagraphStar}[1]{\oldsubparagraph*{#1}\mbox{}}
  \newcommand{\xxxSubParagraphNoStar}[1]{\oldsubparagraph{#1}\mbox{}}
\fi
\makeatother


\usepackage{longtable,booktabs,array}
\usepackage{calc} % for calculating minipage widths
% Correct order of tables after \paragraph or \subparagraph
\usepackage{etoolbox}
\makeatletter
\patchcmd\longtable{\par}{\if@noskipsec\mbox{}\fi\par}{}{}
\makeatother
% Allow footnotes in longtable head/foot
\IfFileExists{footnotehyper.sty}{\usepackage{footnotehyper}}{\usepackage{footnote}}
\makesavenoteenv{longtable}
\usepackage{graphicx}
\makeatletter
\newsavebox\pandoc@box
\newcommand*\pandocbounded[1]{% scales image to fit in text height/width
  \sbox\pandoc@box{#1}%
  \Gscale@div\@tempa{\textheight}{\dimexpr\ht\pandoc@box+\dp\pandoc@box\relax}%
  \Gscale@div\@tempb{\linewidth}{\wd\pandoc@box}%
  \ifdim\@tempb\p@<\@tempa\p@\let\@tempa\@tempb\fi% select the smaller of both
  \ifdim\@tempa\p@<\p@\scalebox{\@tempa}{\usebox\pandoc@box}%
  \else\usebox{\pandoc@box}%
  \fi%
}
% Set default figure placement to htbp
\def\fps@figure{htbp}
\makeatother





\setlength{\emergencystretch}{3em} % prevent overfull lines

\providecommand{\tightlist}{%
  \setlength{\itemsep}{0pt}\setlength{\parskip}{0pt}}



 


\makeatletter
\@ifpackageloaded{tcolorbox}{}{\usepackage[skins,breakable]{tcolorbox}}
\@ifpackageloaded{fontawesome5}{}{\usepackage{fontawesome5}}
\definecolor{quarto-callout-color}{HTML}{909090}
\definecolor{quarto-callout-note-color}{HTML}{0758E5}
\definecolor{quarto-callout-important-color}{HTML}{CC1914}
\definecolor{quarto-callout-warning-color}{HTML}{EB9113}
\definecolor{quarto-callout-tip-color}{HTML}{00A047}
\definecolor{quarto-callout-caution-color}{HTML}{FC5300}
\definecolor{quarto-callout-color-frame}{HTML}{acacac}
\definecolor{quarto-callout-note-color-frame}{HTML}{4582ec}
\definecolor{quarto-callout-important-color-frame}{HTML}{d9534f}
\definecolor{quarto-callout-warning-color-frame}{HTML}{f0ad4e}
\definecolor{quarto-callout-tip-color-frame}{HTML}{02b875}
\definecolor{quarto-callout-caution-color-frame}{HTML}{fd7e14}
\makeatother
\makeatletter
\@ifpackageloaded{caption}{}{\usepackage{caption}}
\AtBeginDocument{%
\ifdefined\contentsname
  \renewcommand*\contentsname{Table of contents}
\else
  \newcommand\contentsname{Table of contents}
\fi
\ifdefined\listfigurename
  \renewcommand*\listfigurename{List of Figures}
\else
  \newcommand\listfigurename{List of Figures}
\fi
\ifdefined\listtablename
  \renewcommand*\listtablename{List of Tables}
\else
  \newcommand\listtablename{List of Tables}
\fi
\ifdefined\figurename
  \renewcommand*\figurename{Figure}
\else
  \newcommand\figurename{Figure}
\fi
\ifdefined\tablename
  \renewcommand*\tablename{Table}
\else
  \newcommand\tablename{Table}
\fi
}
\@ifpackageloaded{float}{}{\usepackage{float}}
\floatstyle{ruled}
\@ifundefined{c@chapter}{\newfloat{codelisting}{h}{lop}}{\newfloat{codelisting}{h}{lop}[chapter]}
\floatname{codelisting}{Listing}
\newcommand*\listoflistings{\listof{codelisting}{List of Listings}}
\usepackage{amsthm}
\theoremstyle{definition}
\newtheorem{example}{Example}[section]
\theoremstyle{definition}
\newtheorem{exercise}{Exercise}[section]
\theoremstyle{plain}
\newtheorem{theorem}{Theorem}[section]
\theoremstyle{remark}
\AtBeginDocument{\renewcommand*{\proofname}{Proof}}
\newtheorem*{remark}{Remark}
\newtheorem*{solution}{Solution}
\newtheorem{refremark}{Remark}[section]
\newtheorem{refsolution}{Solution}[section]
\makeatother
\makeatletter
\makeatother
\makeatletter
\@ifpackageloaded{caption}{}{\usepackage{caption}}
\@ifpackageloaded{subcaption}{}{\usepackage{subcaption}}
\makeatother
\usepackage{bookmark}
\IfFileExists{xurl.sty}{\usepackage{xurl}}{} % add URL line breaks if available
\urlstyle{same}
\hypersetup{
  pdftitle={MATH 557 Oct 15},
  colorlinks=true,
  linkcolor={blue},
  filecolor={Maroon},
  citecolor={Blue},
  urlcolor={Blue},
  pdfcreator={LaTeX via pandoc}}


\title{MATH 557 Oct 15}
\author{}
\date{}
\begin{document}
\maketitle


\section{Amalgamation Classes}\label{amalgamation-classes}

Let \(\overline{K}\) be a class of finitely generated structures.
\(\overline{K}\) is called an \textbf{amalgamation class} if it has the
following three properties:

\begin{tcolorbox}[enhanced jigsaw, toptitle=1mm, left=2mm, toprule=.15mm, rightrule=.15mm, coltitle=black, bottomtitle=1mm, titlerule=0mm, arc=.35mm, colback=white, breakable, title={(HP) Hereditary Property}, bottomrule=.15mm, leftrule=.75mm, opacitybacktitle=0.6, opacityback=0, colbacktitle=quarto-callout-note-color!10!white, colframe=quarto-callout-note-color-frame]

If \(A \in \overline{K}\), \(\mathcal{B} \cong \mathcal{C} \in A\), and
\(\mathcal{C}\) is finitely generated, then
\(\mathcal{B} \in \overline{K}\).

\end{tcolorbox}

\begin{tcolorbox}[enhanced jigsaw, toptitle=1mm, left=2mm, toprule=.15mm, rightrule=.15mm, coltitle=black, bottomtitle=1mm, titlerule=0mm, arc=.35mm, colback=white, breakable, title={(JEP) Joint Embedding Property}, bottomrule=.15mm, leftrule=.75mm, opacitybacktitle=0.6, opacityback=0, colbacktitle=quarto-callout-note-color!10!white, colframe=quarto-callout-note-color-frame]

If \(A, \mathcal{B} \in \overline{K}\), then there exists
\(\mathcal{C} \in \overline{K}\) and embeddings
\[f_0: A \to \mathcal{C}, \quad f_1: \mathcal{B} \to \mathcal{C}\]

\end{tcolorbox}

\begin{tcolorbox}[enhanced jigsaw, toptitle=1mm, left=2mm, toprule=.15mm, rightrule=.15mm, coltitle=black, bottomtitle=1mm, titlerule=0mm, arc=.35mm, colback=white, breakable, title={(AP) Amalgamation Property}, bottomrule=.15mm, leftrule=.75mm, opacitybacktitle=0.6, opacityback=0, colbacktitle=quarto-callout-note-color!10!white, colframe=quarto-callout-note-color-frame]

If \(A, \mathcal{B}, \mathcal{C} \in \overline{K}\) with embeddings
\(f_0: A \to \mathcal{B}\) and \(f_1: A \to \mathcal{C}\), then there
exists \(\mathcal{D} \in \overline{K}\) and embeddings
\[g_0: \mathcal{B} \to \mathcal{D}, \quad g_1: \mathcal{C} \to \mathcal{D}\]
such that \(g_0 \circ f_0 = g_1 \circ f_1\).

\end{tcolorbox}

\begin{exercise}[]\protect\hypertarget{exr-}{}\label{exr-}

\hfill\break
Show that (AP) does not imply (JEP).

(\emph{Hint: finite fields})

\end{exercise}

\begin{example}[]\protect\hypertarget{exm-}{}\label{exm-}

\hfill\break
\(\overline{K} = \{ (Z,<): (Z,<) \text{ finite linear order} \}\) forms
an amalgamation class.

\end{example}

\subsection{Fraïssé's Theorem}\label{frauxefssuxe9s-theorem}

\begin{theorem}[]\protect\hypertarget{thm-fraisse}{}\label{thm-fraisse}

Let \(\overline{K}\) be a class of finitely generated
\(\mathcal{L}\)-structures such that there are only countably many
isomorphism types in \(\overline{K}\).

Then: \(\overline{K}\) is an amalgamation class \(\Leftrightarrow\)
\(\overline{K}\) is the age of a countable homogeneous
\(\mathcal{L}\)-structure.

\end{theorem}

\begin{proof}
\textbf{(\(\Leftarrow\)):} Suppose
\(\overline{K} = \text{age}(\mathcal{M})\) where \(\mathcal{M}\) is
countable and homogeneous.

\textbf{(HP):} Holds by definition of age.

\textbf{(JEP):} Let
\(\mathcal{A}, \mathcal{B} \in \text{age}(\mathcal{M})\). Then
\(A \cong \langle A \rangle^{\mathcal{M}}\) and
\(\mathcal{B} \cong \langle B \rangle^{\mathcal{M}}\), with
\(A,B \subseteq M\) finite. Then \(\mathcal{A}\) and \(\mathcal{B}\)
embed into \(\langle A \cup B \rangle^{\mathcal{M}}\).

\textbf{(AP):} Suppose \(f_0: \mathcal{A} \to \mathcal{B}\) and
\(f_1: \mathcal{A} \to \mathcal{C}\) are embeddings, where
\(\mathcal{A}, \mathcal{B}, \mathcal{C} \in \text{age}(\mathcal{M})\).

Without loss of generality,
\(\mathcal{A}, \mathcal{B}, \mathcal{C} \subseteq \mathcal{M}\) and
\(f_0 = \text{id}_A\).

Since \(\mathcal{M}\) is homogeneous, there exists an automorphism
\(\pi: \mathcal{M} \overset{\sim}{\to} \mathcal{M}\) such that
\(\pi^{-1}|_A = f_1\) (i.e., \(\pi\) extends \(f_1^{-1}\)).

Consider
\(\mathcal{D} = \langle B \cup \pi^{-1}(C) \rangle^{\mathcal{M}}\).
\(\mathcal{D}\) amalgamates \(\mathcal{B}\) and \(\mathcal{C}\) via
\(\text{id}_{\mathcal{B}}: \mathcal{B} \to \mathcal{D}\) and
\(\pi^{-1}|_C: \mathcal{C} \to \mathcal{D}\).
\end{proof}

\textbf{(\(\Rightarrow\)):} Assume \(\overline{K}\) is an amalgamation
class. We construct a structure \(\mathcal{N}\) with
\(\text{age}(\mathcal{N}) = \overline{K}\) where \(\mathcal{N}\) is
homogeneous and has domain \(N \subset \mathbb{N}\).

Since \(\overline{K}\) has countably many isomorphism types, we
enumerate the elements of \(\overline{K}\) (up to isomorphism) as
\((\mathcal{K}_e: e \in \mathbb{N})\) where
\(K_e \subseteq \mathbb{N}\).

Consider tuples \((\bar{a}, \bar{b}, f)\) where \(\bar{a}_i, \bar{b}_i\)
are finite subsets of \(\mathbb{N}\) and \(f\) is a partial function
with \(\operatorname{dom}(f) = \bar{a}\) and
\(\text{ran}(f) \subseteq \bar{b}\). Enumerate all such triples as
\((\bar{a}_e, \bar{b}_e, f_e)\) so that every \((\bar{a}, \bar{b}, f)\)
occurs infinitely often.

\(\mathcal{N}\) will be the union of an increasing sequence
\((\mathcal{C}_n)_{n \in \mathbb{N}}\) where
\(\mathcal{C}_n \in \overline{K}\).

\textbf{Initialize:} \(\mathcal{C}_0 = \mathcal{K}_0\)

Now assume we have defined \(\mathcal{C}_n\).

\textbf{Case \(n = 2\ell\) (even):} Apply (JEP) to
\(\mathcal{C}_n, \mathcal{K}_\ell\) to obtain \(\mathcal{C}_{n+1}\).

\textbf{Case \(n = 2\ell + 1\) (odd):} If \(\bar{a}_\ell\) or
\(\bar{b}_\ell \nsubseteq C_n\), put
\(\mathcal{C}_{n+1} = \mathcal{C}_n\). If
\(\bar{a}_\ell, \bar{b}_\ell \subseteq C_n\), let
\(\mathcal{A}_\ell = \langle \bar{a}_\ell \rangle^{\mathcal{C}_n}\) and
\(\mathcal{B}_\ell = \langle \bar{b}_\ell \rangle^{\mathcal{C}_n} \subseteq \mathcal{C}_n\).
Apply (AP) to the embeddings
\(\operatorname{id}: \mathcal{A}_\ell \to \mathcal{C}_n\) and
\(f: \mathcal{A}_\ell \to \mathcal{B}_\ell\) (induced by
\(f_\ell: \bar{a}_\ell \to \bar{b}_\ell\)). This yields a structure
\(\mathcal{C}_{n+1} \in \overline{K}\). Renaming if necessary, we can
assume \(\mathcal{C}_n \subseteq \mathcal{C}_{n+1}\).

Define \(\mathcal{N} = \bigcup_{n \in \mathbb{N}} \mathcal{C}_n\).

\textbf{Claim 1:} \(\text{age}(\mathcal{N}) = \overline{K}\).\\
In the even steps \(n = 2\ell\), we ensure
\(K_\ell \subseteq \mathcal{N}\), so all structures isomorphic to
\(K_\ell\) also enter \(\text{age}(\mathcal{N})\). Since the
\((K_e)_{e \in \mathbb{N}}\) enumerates all isomorphism types of
\(\overline{K}\), we get \(\text{age}(\mathcal{N}) = \overline{K}\).

\textbf{Claim 2:} \(\mathcal{N}\) is homogeneous.\\
Let \(\tau: A \to \mathcal{B}\) be an isomorphism between finitely
generated substructures
\(\mathcal{A} = \langle \bar{a} \rangle^{\mathcal{N}}\) and
\(\mathcal{B} = \langle \bar{b} \rangle^{\mathcal{N}}\) where
\(\bar{a}, \bar{b} \in N\) are finite. We show that for any
\(c_0 \in N \setminus B\), there exists a partial isomorphism
\(\tau': \mathcal{A}' \to \mathcal{B}'\) where
\(\mathcal{A}', \mathcal{B}'\) are finitely generated,
\(\tau' \supseteq \tau\), and \(c_0 \in \operatorname{dom}(\tau')\).
(This suffices since we can continue via back-and-forth.)

There exists \(\ell\) such that \(f_\ell\) induces an embedding
\(f: A \to \langle \mathcal{B} \cup \{c_0\} \rangle^{\mathcal{N}}\)
(since every triple occurs infinitely often in the enumeration, in
particular the triple \((\bar{a}, \bar{b} \cup \{c_0\}, f)\)).

In step \(n = 2\ell + 1\),
\(\mathcal{B}_\ell =  \in \langle \mathcal{B} \cup \{c_0\} \rangle^{\mathcal{N}}\)
and \(\mathcal{C}_n\) are amalgamated over \(A_\ell\). This yields
embeddings: \[
\begin{aligned}
\operatorname{id}: & \;  \mathcal{A} \hookrightarrow \mathcal{C}_{n+1} \\
 g: & \; \mathcal{B} \hookrightarrow \mathcal{C}_{n+1}  \text{ where } g \circ f_\ell = \operatorname{id}_A \\
\end{aligned}
\] Let \(a_0 = g(c_0)\). Then
\((g|_{\mathcal{B}})^{-1}: \langle \bar{a} \cup \{a_0\} \rangle^{\mathcal{N}} \overset{\sim}{\to} \langle \bar{b} \cup \{c_0\} \rangle^{\mathcal{N}}\)
is the desired isomorphism.




\end{document}
