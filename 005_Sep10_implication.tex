% Options for packages loaded elsewhere
% Options for packages loaded elsewhere
\PassOptionsToPackage{unicode}{hyperref}
\PassOptionsToPackage{hyphens}{url}
\PassOptionsToPackage{dvipsnames,svgnames,x11names}{xcolor}
%
\documentclass[
]{article}
\usepackage{xcolor}
\usepackage[margin=0.8in]{geometry}
\usepackage{amsmath,amssymb}
\setcounter{secnumdepth}{-\maxdimen} % remove section numbering
\usepackage{iftex}
\ifPDFTeX
  \usepackage[T1]{fontenc}
  \usepackage[utf8]{inputenc}
  \usepackage{textcomp} % provide euro and other symbols
\else % if luatex or xetex
  \usepackage{unicode-math} % this also loads fontspec
  \defaultfontfeatures{Scale=MatchLowercase}
  \defaultfontfeatures[\rmfamily]{Ligatures=TeX,Scale=1}
\fi
\usepackage{lmodern}
\ifPDFTeX\else
  % xetex/luatex font selection
\fi
% Use upquote if available, for straight quotes in verbatim environments
\IfFileExists{upquote.sty}{\usepackage{upquote}}{}
\IfFileExists{microtype.sty}{% use microtype if available
  \usepackage[]{microtype}
  \UseMicrotypeSet[protrusion]{basicmath} % disable protrusion for tt fonts
}{}
\makeatletter
\@ifundefined{KOMAClassName}{% if non-KOMA class
  \IfFileExists{parskip.sty}{%
    \usepackage{parskip}
  }{% else
    \setlength{\parindent}{0pt}
    \setlength{\parskip}{6pt plus 2pt minus 1pt}}
}{% if KOMA class
  \KOMAoptions{parskip=half}}
\makeatother
% Make \paragraph and \subparagraph free-standing
\makeatletter
\ifx\paragraph\undefined\else
  \let\oldparagraph\paragraph
  \renewcommand{\paragraph}{
    \@ifstar
      \xxxParagraphStar
      \xxxParagraphNoStar
  }
  \newcommand{\xxxParagraphStar}[1]{\oldparagraph*{#1}\mbox{}}
  \newcommand{\xxxParagraphNoStar}[1]{\oldparagraph{#1}\mbox{}}
\fi
\ifx\subparagraph\undefined\else
  \let\oldsubparagraph\subparagraph
  \renewcommand{\subparagraph}{
    \@ifstar
      \xxxSubParagraphStar
      \xxxSubParagraphNoStar
  }
  \newcommand{\xxxSubParagraphStar}[1]{\oldsubparagraph*{#1}\mbox{}}
  \newcommand{\xxxSubParagraphNoStar}[1]{\oldsubparagraph{#1}\mbox{}}
\fi
\makeatother


\usepackage{longtable,booktabs,array}
\usepackage{calc} % for calculating minipage widths
% Correct order of tables after \paragraph or \subparagraph
\usepackage{etoolbox}
\makeatletter
\patchcmd\longtable{\par}{\if@noskipsec\mbox{}\fi\par}{}{}
\makeatother
% Allow footnotes in longtable head/foot
\IfFileExists{footnotehyper.sty}{\usepackage{footnotehyper}}{\usepackage{footnote}}
\makesavenoteenv{longtable}
\usepackage{graphicx}
\makeatletter
\newsavebox\pandoc@box
\newcommand*\pandocbounded[1]{% scales image to fit in text height/width
  \sbox\pandoc@box{#1}%
  \Gscale@div\@tempa{\textheight}{\dimexpr\ht\pandoc@box+\dp\pandoc@box\relax}%
  \Gscale@div\@tempb{\linewidth}{\wd\pandoc@box}%
  \ifdim\@tempb\p@<\@tempa\p@\let\@tempa\@tempb\fi% select the smaller of both
  \ifdim\@tempa\p@<\p@\scalebox{\@tempa}{\usebox\pandoc@box}%
  \else\usebox{\pandoc@box}%
  \fi%
}
% Set default figure placement to htbp
\def\fps@figure{htbp}
\makeatother





\setlength{\emergencystretch}{3em} % prevent overfull lines

\providecommand{\tightlist}{%
  \setlength{\itemsep}{0pt}\setlength{\parskip}{0pt}}



 


\makeatletter
\@ifpackageloaded{tcolorbox}{}{\usepackage[skins,breakable]{tcolorbox}}
\@ifpackageloaded{fontawesome5}{}{\usepackage{fontawesome5}}
\definecolor{quarto-callout-color}{HTML}{909090}
\definecolor{quarto-callout-note-color}{HTML}{0758E5}
\definecolor{quarto-callout-important-color}{HTML}{CC1914}
\definecolor{quarto-callout-warning-color}{HTML}{EB9113}
\definecolor{quarto-callout-tip-color}{HTML}{00A047}
\definecolor{quarto-callout-caution-color}{HTML}{FC5300}
\definecolor{quarto-callout-color-frame}{HTML}{acacac}
\definecolor{quarto-callout-note-color-frame}{HTML}{4582ec}
\definecolor{quarto-callout-important-color-frame}{HTML}{d9534f}
\definecolor{quarto-callout-warning-color-frame}{HTML}{f0ad4e}
\definecolor{quarto-callout-tip-color-frame}{HTML}{02b875}
\definecolor{quarto-callout-caution-color-frame}{HTML}{fd7e14}
\makeatother
\makeatletter
\@ifpackageloaded{caption}{}{\usepackage{caption}}
\AtBeginDocument{%
\ifdefined\contentsname
  \renewcommand*\contentsname{Table of contents}
\else
  \newcommand\contentsname{Table of contents}
\fi
\ifdefined\listfigurename
  \renewcommand*\listfigurename{List of Figures}
\else
  \newcommand\listfigurename{List of Figures}
\fi
\ifdefined\listtablename
  \renewcommand*\listtablename{List of Tables}
\else
  \newcommand\listtablename{List of Tables}
\fi
\ifdefined\figurename
  \renewcommand*\figurename{Figure}
\else
  \newcommand\figurename{Figure}
\fi
\ifdefined\tablename
  \renewcommand*\tablename{Table}
\else
  \newcommand\tablename{Table}
\fi
}
\@ifpackageloaded{float}{}{\usepackage{float}}
\floatstyle{ruled}
\@ifundefined{c@chapter}{\newfloat{codelisting}{h}{lop}}{\newfloat{codelisting}{h}{lop}[chapter]}
\floatname{codelisting}{Listing}
\newcommand*\listoflistings{\listof{codelisting}{List of Listings}}
\usepackage{amsthm}
\theoremstyle{definition}
\newtheorem{exercise}{Exercise}[section]
\theoremstyle{remark}
\AtBeginDocument{\renewcommand*{\proofname}{Proof}}
\newtheorem*{remark}{Remark}
\newtheorem*{solution}{Solution}
\newtheorem{refremark}{Remark}[section]
\newtheorem{refsolution}{Solution}[section]
\makeatother
\makeatletter
\makeatother
\makeatletter
\@ifpackageloaded{caption}{}{\usepackage{caption}}
\@ifpackageloaded{subcaption}{}{\usepackage{subcaption}}
\makeatother
\usepackage{bookmark}
\IfFileExists{xurl.sty}{\usepackage{xurl}}{} % add URL line breaks if available
\urlstyle{same}
\hypersetup{
  pdftitle={Math 557 Sep 10},
  colorlinks=true,
  linkcolor={blue},
  filecolor={Maroon},
  citecolor={Blue},
  urlcolor={Blue},
  pdfcreator={LaTeX via pandoc}}


\title{Math 557 Sep 10}
\author{}
\date{}
\begin{document}
\maketitle


\section{Logical Implication and
Proof}\label{logical-implication-and-proof}

\subsection{Key Concepts}\label{key-concepts}

\begin{itemize}
\item
  \textbf{Logical consequence}:

  \begin{itemize}
  \tightlist
  \item
    This is the semantical implication we are often working with in
    mathematical practice. We say \(T\) \textbf{logically implies}
    \(\varphi\), \(T \models \varphi\), if for structure
    \(\mathcal{M}\), \(\mathcal{M} \models T\) implies
    \(\mathcal{M} \models \varphi\).
  \end{itemize}
\item
  \textbf{Formal proof}:

  \begin{itemize}
  \tightlist
  \item
    \(T \vdash \varphi\) means there is a formal
    (i.e.~\emph{syntactical}) derivation of \(\varphi\) from \(T\) using
    the formulas of \(T\), the three kinds of \emph{logical axioms}
    (propositional tautologies, equality and quantifier axioms), and the
    \emph{inference rules} Modus Ponens and Generalization.
  \end{itemize}
\end{itemize}

\subsection{Problems}\label{problems}

\begin{exercise}[Warmup - Logical
Implication]\protect\hypertarget{exr-}{}\label{exr-}

\hfill\break
Let \(T\) be an \(\mathcal{L}\)-theory. We say a theory \(T'\) is an
\textbf{axiomatization} of \(T\) if for any \(\mathcal{L}\)-structure
\(\mathcal{M}\),

\[\mathcal{M} \models T \; \iff \; \mathcal{M} \models T'\]

Show that for any axiomatization \(T'\) of \(T\), for any
\(\mathcal{L}\)-sentence \(\sigma\),

\[T \models \sigma \; \iff \; T' \models \sigma\]

\end{exercise}

\begin{exercise}[Warmup 2]\protect\hypertarget{exr-}{}\label{exr-}

\hfill\break
Recall that a \emph{model} of a theory \(T\) is a structure
\(\mathcal{M}\) such that for any sentence \(\sigma \in T\),
\(\mathcal{M} \models \sigma\). In this case we write
\(\mathcal{M} \models T\).

Argue that if \(T\) does not have a model, every sentence is a logical
implication of \(T\).

\end{exercise}

\begin{exercise}[Formal notion of proof --
Warmup]\protect\hypertarget{exr-}{}\label{exr-}

\hfill\break
Verify that

\[\{\varphi, \neg \psi\} \vdash \neg(\varphi \to \psi)\]

\end{exercise}

\begin{exercise}[]\protect\hypertarget{exr-}{}\label{exr-}

\hfill\break
Argue (semantically) that if \(x\) is not free in \(\psi\),

\[\{\varphi \to \psi\} \models \exists x \varphi \: \to \: \psi\]

Then prove this \emph{syntactically}, i.e.~show (under the same
assumption) that

\[\{\varphi \to \psi\} \vdash \exists x \varphi \: \to \: \psi\]

\end{exercise}

\begin{exercise}[]\protect\hypertarget{exr-}{}\label{exr-}

\hfill\break
Prove the \emph{Soundness Theorem}, i.e.~show that

\[T \vdash \varphi \; \Rightarrow \; T \models \varphi\]

\end{exercise}

\begin{tcolorbox}[enhanced jigsaw, opacityback=0, breakable, bottomrule=.15mm, toprule=.15mm, toptitle=1mm, colframe=quarto-callout-important-color-frame, left=2mm, leftrule=.75mm, coltitle=black, colbacktitle=quarto-callout-important-color!10!white, opacitybacktitle=0.6, bottomtitle=1mm, rightrule=.15mm, arc=.35mm, title=\textcolor{quarto-callout-important-color}{\faExclamation}\hspace{0.5em}{Take-home problem}, colback=white, titlerule=0mm]

\hfill\break
Show that

\begin{align*}
\{\varphi \to \psi \} & \vdash \exists x \varphi \: \to \: \exists x \psi \\
\{\varphi \to \psi \} & \vdash \forall x \varphi \: \to \: \forall x \psi \\
\end{align*}

\end{tcolorbox}




\end{document}
