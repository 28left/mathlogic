Für Terme $t$ und Formeln $\varphi$ der Sprache von  $PA^-$ schreiben wir
\begin{eqnarray*}
\exists x < t \;  \varphi & \text{für} & \exists x (x < t \wedge \varphi)\\
\forall x < t \;  \varphi & \text{für} & \forall x (x < t \to \varphi)
\end{eqnarray*}
\noindent und nennen $\exists x < t$  und $\forall x < t$ {\bf beschränkte} Quantoren.

\subsection*{Definition}
\begin{tabbing}
\qquad \= $\varphi$ ist $\Delta_0$-Formel: \; \= $\iff \varphi$ enthält höchstens beschränkte Quantoren,\\
\> $\varphi$ ist $\Sigma_1$-Formel: \; \> $\iff \varphi = \exists \vec{x} \; \psi$ \; für eine $\Delta_0$-Formel $\psi$,\\
\> $\varphi$ ist $\Pi_1$-Formel: \; \>  $\iff \varphi = \forall \vec{x}  \;\psi$ \; für eine $\Delta_0$-Formel $\psi$,
\end{tabbing}
\noindent Dieses ist der Anfang der {\bf Hierarchie der arithmetischen Formeln}; setzt man
\begin{equation*}
\Sigma_0 = \Pi_0 = \Delta_0,
\end{equation*}
so kann man fortsetzen:

\begin{tabbing}
\qquad \quad \= $\varphi$ ist $\Delta_0$-Formel \quad  \=  $\iff \varphi$ höchstens beschränkte Quantoren enthält,\kill
\> $\varphi$ ist $\Sigma_{n+1}$-Formel \; \>  $\iff \varphi = \exists \vec{x} \; \psi$ \; für eine $\Pi_n$-Formel $\psi$,\\
\> $\varphi$ ist $\Pi_{n+1}$-Formel \; \>  $\iff \varphi = \forall \vec{x} \; \psi$ \; für eine $\Pi_n$-Formel $\psi$.
\end{tabbing}

Eine $\Sigma_3$-Formel ist also von der Form $\exists \vec{x} \; \forall \vec{y}\; \exists \vec{z} \; \psi$, wobei $\psi$ höchstens beschränkte Quantoren enthält. Das bedeutet also, dass man beschränkte Quantoren nicht mitzählt, mit $\Sigma$ bzw. $\Pi$ anzeigt, dass die Formel mit einer (endlichen) Folge von $\exists$-Quantoren bzw. $\forall$-Quantoren beginnt und der Index die Quantorenblöcke zählt, so dass es weniger auf die Anzahl der Quantoren als auf die Anzahl der Quantorenwechsel ankommt.

Bei dieser Klassifizierung unterscheidet man nicht zwischen logisch äquivalenten Formeln, so dass etwa jede $\Pi_n$-Formel für $n<m$ auch eine $\Sigma_m$- und $\Pi_m$-Formel ist (indem man der Formel einfach zusätzliche Quantoren über nicht vorkommende Variablen voranstellt). Somit kann man dann auch die Formelmengen

\begin{equation*}
 \Delta_n = \Sigma_n \cap \Pi_n
\end{equation*}

\noindent definieren. Es ergibt sich daraus folgendes Bild der {\bf arithmetischen Hierarchie}:

\setlength{\unitlength}{1cm}
\begin{picture}(12,5)\thicklines

\put(0.7,3){\makebox(0,0)[t]{$\Delta_0$}}
\put(1.1,2.8){\line(1,0){0.6}}
\put(2.1,3){\makebox(0,0)[t]{$\Delta_1$}}

\put(2.2,3){\line(1,1){0.8}}
\put(2.2,2.6){\line(1,-1){0.8}}

\put(3.4,2){\makebox(0,0)[t]{$\Pi_1$}}
\put(3.4,4){\makebox(0,0)[t]{$\Sigma_1$}}

\put(3.8,3.8){\line(1,-1){0.8}}
\put(3.8,1.8){\line(1,1){0.8}}

\put(4.9,3){\makebox(0,0)[t]{$\Delta_2$}}

\put(5.1,3){\line(1,1){0.8}}
\put(5.1,2.6){\line(1,-1){0.8}}

\put(6.3,2){\makebox(0,0)[t]{$\Pi_2$}}
\put(6.3,4){\makebox(0,0)[t]{$\Sigma_2$}}

\put(6.6,3.8){\line(1,-1){0.8}}
\put(6.6,1.8){\line(1,1){0.8}}

\put(7.6,3){\makebox(0,0)[t]{$\Delta_3$}}

\put(7.8,3){\line(1,1){0.8}}
\put(7.8,2.6){\line(1,-1){0.8}}

\put(9.6,2.8){\makebox(0,0)[t]{\ldots \ldots}}

\end{picture}
  \hspace*{1cm}

Viele nützliche Eigenschaften natürlicher Zahlen lassen sich mittels $\Delta_0$-For\-meln ausdrücken, z.\, B.:

\begin{equation*}
x \; ist \; irreduzibel   \iff 1 < x \wedge \forall u<x \; \forall v<x \; \neg (u \cdot v = x).
\end{equation*}



Hier soll gezeigt werden, dass die rekursive Relationen mit den Mengen übereinstimmen, die sich durch eine $\Delta_1$-Formeln in den natürlichen Zahlen definieren lassen, und dass der Graph einer  rekursiven Funktion durch eine  $\Sigma_1$-Formeln in den natürlichen Zahlen definiert werden kann. Wir beginnen mit dem

\section{Lemma über $\Delta_0$-Formeln} \label{delta0formeln}
\emph{Für jede $\Delta_0$-Formel $\theta(\vec{v})$ ist die Relation}

\begin{equation*}
R(\vec{a}) \iff \mathbb{N} \models \theta(\vec{a})
\end{equation*}
  \emph{primitiv-rekursiv.}
  
  \begin{Beweis}
Wir zeigen durch Induktion über lz($\theta$), dass die zugehörige charakteristische Funktion

\medskip

$c_{\theta}(\vec{x}) = 
\begin{cases}
    1  &  \text{falls } \mathbb{N} \models \theta(\vec{x})\\
    0   &   \text{sonst}
\end{cases}$

\medskip

\noindent primitiv-rekursiv ist: Zunächst sind die Funktionen $x+1, x+y, x \cdot y$ p.r. und damit definiert jeder Term in $\mathbb{N}$ eine primitiv-rekursive Funktion. Da auch die Funktionen $eq(x,y)=\overline{sg}(|x-y|)$ und $sg(y \Pmin x)$ primitv-rekursiv sind (und die p.r. Funktionen abgeschlossen sind unter Substitution), gilt die Behauptung für die atomaren Formeln $t=s, t < s$. Für den Fall der aussagenlogischen Operationen  benutze man
\begin{equation*}
c_{\neg \theta}(\vec{x}) = 1 \Pmin c_{\theta}(\vec{x}),  \; c_{\theta \wedge \psi}(\vec{x}) = c_{\theta}(\vec{x}) \cdot c_{\psi}(\vec{x}), \; c_{\theta \vee \psi}(\vec{x}) = \min(c_{\theta}(\vec{x}),c_{\psi}(\vec{x})).
\end{equation*}
Ist schließlich $\theta$ eine $\Delta_0$-Formel, $t$ ein Term und $\psi(\vec{x}) = \forall y < t(\vec{x}) \; \theta(\vec{x},y)$, so folgt die Behauptung aus

\begin{equation*}
c_{ \psi}(\vec{x}) = eq( t(\vec{x}) , (\mu y \le t(\vec{x}) (c_{\theta}(\vec{x},y) = 0)).
\end{equation*}

\noindent Ähnlich argumentiert man im Falle der Formel  $ \exists y < t(\vec{x}) \; \theta(\vec{x},y)$ (oder führt diesen Fall mittels der Negation auf den früheren zurück).
\end{Beweis}

Die Umkehrung des obigen Lemmas gilt nicht: es gibt primitiv-rekursive Mengen, die durch keine $\Delta_0$-Formel in den natürlichen Zahlen definierbar sind. 
