% Options for packages loaded elsewhere
% Options for packages loaded elsewhere
\PassOptionsToPackage{unicode}{hyperref}
\PassOptionsToPackage{hyphens}{url}
\PassOptionsToPackage{dvipsnames,svgnames,x11names}{xcolor}
%
\documentclass[
]{article}
\usepackage{xcolor}
\usepackage[margin=1in]{geometry}
\usepackage{amsmath,amssymb}
\setcounter{secnumdepth}{-\maxdimen} % remove section numbering
\usepackage{iftex}
\ifPDFTeX
  \usepackage[T1]{fontenc}
  \usepackage[utf8]{inputenc}
  \usepackage{textcomp} % provide euro and other symbols
\else % if luatex or xetex
  \usepackage{unicode-math} % this also loads fontspec
  \defaultfontfeatures{Scale=MatchLowercase}
  \defaultfontfeatures[\rmfamily]{Ligatures=TeX,Scale=1}
\fi
\usepackage{lmodern}
\ifPDFTeX\else
  % xetex/luatex font selection
\fi
% Use upquote if available, for straight quotes in verbatim environments
\IfFileExists{upquote.sty}{\usepackage{upquote}}{}
\IfFileExists{microtype.sty}{% use microtype if available
  \usepackage[]{microtype}
  \UseMicrotypeSet[protrusion]{basicmath} % disable protrusion for tt fonts
}{}
\makeatletter
\@ifundefined{KOMAClassName}{% if non-KOMA class
  \IfFileExists{parskip.sty}{%
    \usepackage{parskip}
  }{% else
    \setlength{\parindent}{0pt}
    \setlength{\parskip}{6pt plus 2pt minus 1pt}}
}{% if KOMA class
  \KOMAoptions{parskip=half}}
\makeatother
% Make \paragraph and \subparagraph free-standing
\makeatletter
\ifx\paragraph\undefined\else
  \let\oldparagraph\paragraph
  \renewcommand{\paragraph}{
    \@ifstar
      \xxxParagraphStar
      \xxxParagraphNoStar
  }
  \newcommand{\xxxParagraphStar}[1]{\oldparagraph*{#1}\mbox{}}
  \newcommand{\xxxParagraphNoStar}[1]{\oldparagraph{#1}\mbox{}}
\fi
\ifx\subparagraph\undefined\else
  \let\oldsubparagraph\subparagraph
  \renewcommand{\subparagraph}{
    \@ifstar
      \xxxSubParagraphStar
      \xxxSubParagraphNoStar
  }
  \newcommand{\xxxSubParagraphStar}[1]{\oldsubparagraph*{#1}\mbox{}}
  \newcommand{\xxxSubParagraphNoStar}[1]{\oldsubparagraph{#1}\mbox{}}
\fi
\makeatother


\usepackage{longtable,booktabs,array}
\usepackage{calc} % for calculating minipage widths
% Correct order of tables after \paragraph or \subparagraph
\usepackage{etoolbox}
\makeatletter
\patchcmd\longtable{\par}{\if@noskipsec\mbox{}\fi\par}{}{}
\makeatother
% Allow footnotes in longtable head/foot
\IfFileExists{footnotehyper.sty}{\usepackage{footnotehyper}}{\usepackage{footnote}}
\makesavenoteenv{longtable}
\usepackage{graphicx}
\makeatletter
\newsavebox\pandoc@box
\newcommand*\pandocbounded[1]{% scales image to fit in text height/width
  \sbox\pandoc@box{#1}%
  \Gscale@div\@tempa{\textheight}{\dimexpr\ht\pandoc@box+\dp\pandoc@box\relax}%
  \Gscale@div\@tempb{\linewidth}{\wd\pandoc@box}%
  \ifdim\@tempb\p@<\@tempa\p@\let\@tempa\@tempb\fi% select the smaller of both
  \ifdim\@tempa\p@<\p@\scalebox{\@tempa}{\usebox\pandoc@box}%
  \else\usebox{\pandoc@box}%
  \fi%
}
% Set default figure placement to htbp
\def\fps@figure{htbp}
\makeatother





\setlength{\emergencystretch}{3em} % prevent overfull lines

\providecommand{\tightlist}{%
  \setlength{\itemsep}{0pt}\setlength{\parskip}{0pt}}



 


\makeatletter
\@ifpackageloaded{caption}{}{\usepackage{caption}}
\AtBeginDocument{%
\ifdefined\contentsname
  \renewcommand*\contentsname{Table of contents}
\else
  \newcommand\contentsname{Table of contents}
\fi
\ifdefined\listfigurename
  \renewcommand*\listfigurename{List of Figures}
\else
  \newcommand\listfigurename{List of Figures}
\fi
\ifdefined\listtablename
  \renewcommand*\listtablename{List of Tables}
\else
  \newcommand\listtablename{List of Tables}
\fi
\ifdefined\figurename
  \renewcommand*\figurename{Figure}
\else
  \newcommand\figurename{Figure}
\fi
\ifdefined\tablename
  \renewcommand*\tablename{Table}
\else
  \newcommand\tablename{Table}
\fi
}
\@ifpackageloaded{float}{}{\usepackage{float}}
\floatstyle{ruled}
\@ifundefined{c@chapter}{\newfloat{codelisting}{h}{lop}}{\newfloat{codelisting}{h}{lop}[chapter]}
\floatname{codelisting}{Listing}
\newcommand*\listoflistings{\listof{codelisting}{List of Listings}}
\usepackage{amsthm}
\theoremstyle{definition}
\newtheorem{exercise}{Exercise}[section]
\theoremstyle{remark}
\AtBeginDocument{\renewcommand*{\proofname}{Proof}}
\newtheorem*{remark}{Remark}
\newtheorem*{solution}{Solution}
\newtheorem{refremark}{Remark}[section]
\newtheorem{refsolution}{Solution}[section]
\makeatother
\makeatletter
\makeatother
\makeatletter
\@ifpackageloaded{caption}{}{\usepackage{caption}}
\@ifpackageloaded{subcaption}{}{\usepackage{subcaption}}
\makeatother
\usepackage{bookmark}
\IfFileExists{xurl.sty}{\usepackage{xurl}}{} % add URL line breaks if available
\urlstyle{same}
\hypersetup{
  pdftitle={Math 557 Sep 5},
  colorlinks=true,
  linkcolor={blue},
  filecolor={Maroon},
  citecolor={Blue},
  urlcolor={Blue},
  pdfcreator={LaTeX via pandoc}}


\title{Math 557 Sep 5}
\author{}
\date{}
\begin{document}
\maketitle


\section{Substitution}\label{substitution}

\subsection{Key Concepts}\label{key-concepts}

\begin{itemize}
\item
  \textbf{Substitution}:

  \begin{itemize}
  \tightlist
  \item
    Basic idea: \(\varphi_{\bar{s}/\bar{x}}\) is obtained by replacing
    all occurences of the variable \(x_i\) by the term \(s_i\).
  \item
    Uncontrolled substitution may cause issues with quatifiers. If we
    try to substitute a variable into the range of a quantifier, we
    rename the quantified variable to an unused variable
    (\(\exists x \dots\) becomes \(\exits u \dots\)).
  \end{itemize}
\item
  \textbf{Substitution Lemma}:

  \begin{itemize}
  \tightlist
  \item
    Substitution behaves ``as expected'' with respect to evalution and
    satisfaction.
  \item
    Evaluating a substituted term yields the same value as evaluating
    the original term under the ``substituted'' assignment (i.e.~the
    assignment in which we replace the assignment to \(x\) by the value
    of \(s\) under \(\alpha\)).
  \item
    A substituted formula holds in \(\mathcal{M}\) under assignment
    \(\alpha\) iff the original formula holds in \(\mathcal{M}\) under
    the ``substituted'' assignment.
  \end{itemize}
\end{itemize}

\subsection{Problems}\label{problems}

\begin{exercise}[Carry-over from Sep
3]\protect\hypertarget{exr-}{}\label{exr-}

\hfill\break
Show that if \(x\) is not free in \(\varphi\),
\(\mathcal{M} \models \varphi[\alpha]\) implies
\(\mathcal{M} \models \forall x \, \varphi [\alpha]\).

Then verify that

\[\forall x ( \varphi \to \psi) \; \to \; (\varphi \to \forall x \psi) \quad (\text{$x$ not free in $\varphi$})\]

is a validity.

\end{exercise}

\begin{exercise}[]\protect\hypertarget{exr-}{}\label{exr-}

\hfill\break
- Show that if \(t\) is a term, then \(T_{\bar{s}/\bar{x}}\) is a term.

\begin{itemize}
\tightlist
\item
  Show that if \(\varphi\) is a formula, \(\varphi_{\bar{s}/\bar{x}}\)
  is a formula of the same height.
\end{itemize}

\end{exercise}

\begin{exercise}[]\protect\hypertarget{exr-}{}\label{exr-}

\hfill\break
Use the Substitution Lemma to verify that
\(\varphi_{x/t} \; \to \; \exists x \, \varphi\) is a validity.

\end{exercise}

\begin{exercise}[]\protect\hypertarget{exr-}{}\label{exr-}

\hfill\break
Show that if \(y\) does not occur in \(\psi\),

\[
[\psi_{y/x}]_{x/y}
\]

Find a counterexample that shows this no longer holds if \(y\)
\emph{does} occur in \(\psi\).

\end{exercise}




\end{document}
